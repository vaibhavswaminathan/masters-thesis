%%%%%%%%%%%%%%%%%%%%%%%%%%%%%%%%%%%%%%%%%%%%%%%%%%%%%%%%%%%%%
%% HEADER
%%%%%%%%%%%%%%%%%%%%%%%%%%%%%%%%%%%%%%%%%%%%%%%%%%%%%%%%%%%%%
%%
%%
\newcommand{\hyperrefpdfauthor}{}
\newcommand{\hyperrefpdftitle}{}
\newcommand{\hyperrefpdfsubject}{}
\newcommand{\hyperrefpdfkeywords}{}
\newcommand{\hyperrefpdfborder}{0}
\documentclass[master]{wissdoc-kw-eng} % possible options: bachelor, master, diploma, phdsubmitted (vorgelegte Dissertation), phdapproved (genehmigte Dissertation)
% additional logo options: fkie (for Fraunhofer FKIE logo on the left), iop (to use a combiend COMSYS and Internet of Production logo on the right)

\usepackage{pdfpages}

\usepackage{acro}

\makeatletter
\@ifpackagelater{acro}{2017/02/17}{\@ifpackagelater{acro}{2019/09/23}{}{
	\let\IDeclareAcronym\DeclareAcronym
	\renewcommand{\DeclareAcronym}[2]{%
		\IDeclareAcronym{#1}{%
		#2,foreign-plural={}
		}
	}
}}{}
\makeatother

\newcommand\ifacrothree{%
	\ifnum\numexpr\csname c_acro_version_major_number_tl\endcsname=3\relax
}

\ifacrothree
	\acsetup{use-id-as-short}
	\def\ProvideAcroEnding{\DeclareAcroEnding}
\fi

\ProvideAcroEnding{possessive}{'s}{'s}
\ProvideAcroEnding{possessiveplural}{s'}{s'}

\ifacrothree
	\NewAcroCommand\acg{m}{%
		\acropossessive
		\UseAcroTemplate{first}{#1}%
	}
	\NewAcroCommand\acsg{m}{%
		\acropossessive
		\UseAcroTemplate{short}{#1}%
	}
	\NewAcroCommand\aclg{m}{%
		\acropossessive
		\UseAcroTemplate{long}{#1}%
	}
	\NewAcroCommand\acgp{m}{%
		\acropossessiveplural
		\UseAcroTemplate{first}{#1}%
	}
	\NewAcroCommand\acsgp{m}{%
		\acropossessiveplural
		\UseAcroTemplate{short}{#1}%
	}
	\NewAcroCommand\aclgp{m}{%
		\acropossessiveplural
		\UseAcroTemplate{long}{#1}%
	}
\else
	% remove this part when acro-2 (texlive-2019) is no longer in use by any sane person
	% e.g. ubuntu-20.04 has texlive-2019 and will receive updates until April 2025
	\ExplSyntaxOn
	\NewAcroCommand \acg {
		\acro_possessive:
		\acro_use:n {#1}
	}
	\NewAcroCommand \acsg {
		\acro_possessive:
		\acro_short:n {#1}
	}
	\NewAcroCommand \aclg {
		\acro_possessive:
		\acro_long:n {#1}
	}
	\NewAcroCommand \acgp {
		\acro_possessiveplural:
		\acro_use:n {#1}
	}
	\NewAcroCommand \acsgp {
		\acro_possessiveplural:
		\acro_short:n {#1}
	}
	\NewAcroCommand \aclgp {
		\acro_possessiveplural:
		\acro_long:n {#1}
	}
	\ExplSyntaxOff
\fi

% By default, LaTeX uses the acronym name as its short form
\DeclareAcronym{LP}{short=LP,long=Logical Process,short-indefinite=an,long-plural=es}
% You should explicitly set the short form (e.g., if you want lowercase acronym names) to avoid a warning,
% in this case you MUST specify the short= keyword as the first one.
% Otherwise, the document doesen't currently compile under MacTeX (at least).
\DeclareAcronym{abc}{short=ABC,long=Alphabet,short-indefinite=an,long-indefinite=an}



\usepackage{multibib}
\newcites{own}{Pre-published Papers}

% % newer acronym packaged dont use bflabel anymore,
\providecommand\aclabelfont{} % the new command ist aclabelfont however, we also need to be backwards compatible so... we do it like this
\renewcommand{\aclabelfont}[1]{\normalfont{\normalsize{#1}}\hfill} % keine serifenlose schrift für acronym

% newer acronym packaged dont use bflabel anymore, to not fail the statement below provide the command if not existing
\providecommand\bflabel{}
\renewcommand{\bflabel}[1]{\aclabelfont{#1}}

\usepackage{float}
\floatstyle{ruled}
\newfloat{listing}{htbp}{lop}[chapter]
\floatname{listing}{Listing}

%% zur Zitaten des Quelltextes%%%%%%%%%%%%%%%%%%%%%%%%%%%%%%%
% "final" forces printing of all listings, even if the global "draft" is set
\usepackage[final]{listings}
\lstset{
    language=C++,
    basicstyle=\footnotesize\ttfamily,
    tabsize=2,
    numberstyle=\tiny\color{gray},
    numbersep=5pt,
    numbers=left,
    captionpos=b,
    abovecaptionskip=0pt,
    belowcaptionskip=0pt,
    aboveskip=10pt,
    belowskip=0pt,
    floatplacement=tbp,
    frame=topline,
    framerule=.1pt,
    framesep = 3pt,
    mathescape=true,
    boxpos=t,
    xleftmargin=1.2em,
}

\renewcommand\lstlistingname{\textbf{Listing}}
% This is only kept for backwards compatibility. You should never have to use it. Use the listing-environment instead.
%\DeclareCaptionFormat*{lstruled}{{\bfseries#1\small\space\normalfont#3\hrule height.1pt depth0pt}\par}
%\captionsetup[lstlisting]{format=lstruled,singlelinecheck=false}

%% Stil der Beschriftungen %%%%%%%%%%%%%%%%%%%%%%%%%%%%%%%%%%
\usepackage{caption}
\captionsetup[figure]{font={small,sf}}
\captionsetup[subfigure]{justification=centering,font={small,sf}}
\captionsetup[table]{font={small,sf}}
\captionsetup[listing]{font={small,sf}}

%% Normales LaTeX oder pdfLaTeX? %%%%%%%%%%%%%%%%%%%%%%%%%%%%
%% ==> Das neue if-Kommando "\ifpdf" wird an einigen wenigen
%% ==> Stellen benötigt, um die Kompatibilität zwischen
%% ==> LaTeX und pdfLaTeX herzustellen.
%\newif\ifpdf
%\ifx\pdfoutput\undefined
%    \pdffalse              %%normales LaTeX wird ausgeführt
%\else
%    \pdfoutput=1
%    \pdftrue               %%pdfLaTeX wird ausgeführt
%\fi

%% Deutsche Anpassungen %%%%%%%%%%%%%%%%%%%%%%%%%%%%%%%%%%%%%
\usepackage[T1]{fontenc}
\usepackage[utf8]{inputenc}

%% mehrere Abbildungen in eine %%%%%%%%%%%%%%%%%%%%%%%%%%%%%%
\usepackage{subcaption}

%% Packages für Formeln %%%%%%%%%%%%%%%%%%%%%%%%%%%%%%%%%%%%%
\usepackage{amsmath}
\usepackage{amsfonts}

\usepackage[inline]{enumitem}
\setlist[description]{font=\sffamily\bfseries}
\usepackage[capitalise,nameinlink,noabbrev]{cleveref}

%% Zeilenabstand %%%%%%%%%%%%%%%%%%%%%%%%%%%%%%%%%%%%%%%%%%%%
\usepackage{setspace}
%\singlespacing        %% 1-zeilig (Standard)
%\onehalfspacing       %% 1,5-zeilig
%\doublespacing        %% 2-zeilig

%% Andere Packages %%%%%%%%%%%%%%%%%%%%%%%%%%%%%%%%%%%%%%%%%%
%\usepackage{a4wide} %%Kleinere Seitenränder = mehr Text pro Zeile.
\usepackage{fancyhdr} %%Fancy Kopf- und Fußzeilen
%\usepackage{longtable} %%Für Tabellen, die eine Seite überschreiten
\usepackage{lscape}
\usepackage{rotating}
%\usepackage[htt]{hyphenat} %Trennung von Typewriter-Schriften
%\usepackage{listings}
\usepackage{tikz}
\usetikzlibrary{calc}
\usepackage[en-US]{datetime2} %%Automatisches Ausfüllen der eidesstattlichen Versicherung

% Tabellen mit Center und left
\usepackage{tabularx,colortbl} % colored table background
\newcolumntype{C}[1]{>{\centering\arraybackslash}p{#1}}
\newcolumntype{R}[1]{>{\raggedleft\arraybackslash}p{#1}}
% Table spacings
\newcommand\T{\rule{0pt}{2.5ex}\rule[-1.0ex]{0pt}{0pt}}
\newcommand\B{\rule[-1.0ex]{0pt}{0pt}}

\definecolor{slightgray}{gray}{.90}


% Put your own defs to definitions.tex, so you don't need to fiddle around with this file.
\IfFileExists{definitions.tex}{\input{definitions.tex}}{}

%% zur Benutzung bei ergänzenden Daten%%%%%%%%%%%%%%%%%%%%%%%%
%\usepackage{endnotes}
%\renewcommand{\notesname}{Konfigurationsdaten der Messreihen}
%\renewcommand{\theendnote}{\Alph{endnote}}
%\renewcommand{\enotesize}{\normalsize}

%\hyphenation{Sensor-netz-werk
%}

%%%%%%%%%%%%%%%%%%%%%%%%%%%%%%%%%%%%%%%%%%%%%%%%%%%%%%%%%%%%%
%% DOKUMENT
%%%%%%%%%%%%%%%%%%%%%%%%%%%%%%%%%%%%%%%%%%%%%%%%%%%%%%%%%%%%%

% Use the following for selectively updating only individual
% chapters. Multiple chapters can be included as a comma-separated
% list.

% \includeonly{chapters/00_introduction.tex}

\begin{document}

%% Dateiendungen für Grafiken %%%%%%%%%%%%%%%%%%%%%%%%%%%%%%%
%% ==> Sie können hiermit die Dateiendung einer Grafik weglassen.
%% ==> Aus "\includegraphics{titel.eps}" wird "\includegraphics{titel}".
%% ==> Wenn Sie nunmehr 2 inhaltsgleiche Grafiken "titel.eps" und
%% ==> "titel.pdf" erstellen, wird jeweils nur die Grafik eingebunden,
%% ==> die von ihrem Compiler verarbeitet werden kann.
%% ==> pdfLaTeX benutzt "titel.pdf". LaTeX benutzt "titel.eps".
%\ifpdf
%    \DeclareGraphicsExtensions{.pdf,.jpg,.png}
%\else
%    \DeclareGraphicsExtensions{.eps}
%\fi

\pagestyle{empty} %%Keine Kopf-/Fusszeilen auf den ersten Seiten.

\ifnotdraft{
%% Deckblatt %%%%%%%%%%%%%%%%%%%%%%%%%%%%%%%%%%%%%%%%%%%%%%%%
\frontmatter
% !TEX root = ../thesis.tex

\newcommand{\thesistitleA}{% Note the %-symbol at end of the line of each command
Your Awesome Thesis Title%
}
\newcommand{\thesistitleB}{%
(Which May Also Be Quite Long%
}
\newcommand{\thesistitleC}{%
And Stretch Several Lines) Here%
}
\newcommand{\firstname}{%
Your Firstname(s)%
}
\newcommand{\lastname}{%
Your Lastname(s)%
}
\newcommand{\studentid}{%
XXX XXX % This is optional, remove text if you do not want to specific your student id
}
\newcommand{\currentdegree}{%
Master of Science % only for PhD thesis, typically your current degree is Master of Science if you're writing your PhD thesis
}
\newcommand{\aimeddegreeGermanPossesive}{% only for PhD thesis. Degree you are about to get in German, possesive case, including indefinite article
% reasonable options: eines Doktors der Naturwissenschaften, einer Doktorin der Naturwissenschaften, eines Doktors der Ingenieurwissenschaften, einer Doktorin der Ingenieurwissenschaften
eines Doktors der Naturwissenschaften%
}
\newcommand{\placeofbirth}{%
Your Place of Birth % only for PhD thesis, include land if land!=Germany
}
\DTMsavedate{RegistrationDate}{2019-01-01} % only undergrad
\DTMsavedate{SubmissionDate}{2019-08-30} % undergrad and phd
\DTMsavedate{DefenseDate}{2019-08-30} % only phdapproved

\newcommand{\advisers}{% only undergrad
Dipl.-Inform. Erika Mustermann\\
Max Mustermann, M.$\,$Sc.%
}
\newcommand{\examiners}{%
Prof.~Dr.-Ing. Klaus Wehrle\\
Prof.~Dr.~rer.nat. Albert Einstein%
}



% Usually, you do NOT have to change information below.

\begin{titlepage}
\pdfbookmark[0]{Title Page}{Title Page}

% for PhD theses this needs to conform with the chair regulations (Promotions-Merkblatt komplett, Homepage Promotionsbuero Fakultät 1, RWTH Aachen)!

\ifundergrad
	\ifiop
		\tikz[overlay,remember picture]\node[inner sep=0,anchor=north east] at ($(current page.north east)+(2cm,1.7cm)$) {%
			\includegraphics[width=15cm,keepaspectratio]{logos/comsys-iop-rwth}%
		};
	\else
		\tikz[overlay,remember picture]\node[inner sep=0,anchor=north east] at ($(current page.north east)+(2.5cm,2.5cm)$) {%
			\includegraphics[height=3.2cm,keepaspectratio]{logos/rwth_comsys_bild_cmyk}%
		};
	\fi
	\iffkie
		\tikz[overlay,remember picture]\node[inner sep=0,anchor=north west] at ($(current page.north west)+(4cm,1.7cm)$) {%
			\includegraphics[width=6cm,keepaspectratio]{logos/fkie_60mm_p334}%
		};
	\fi
\fi

\hbox{}
\vfill % vertically center text on page

\centering

\ifundergrad
	\def\titleformat{\huge\sffamily}
\fi
\ifphd
	\def\titleformat{\LARGE}
\fi

{\titleformat\textbf{%
\phantom{g}\thesistitleA\phantom{g}\\[.1em]
\phantom{g}\thesistitleB\phantom{g}\\[.1em]
\phantom{g}\thesistitleC\phantom{g}\\[.1em]
}}

\vskip 2cm

\ifundergrad
{\large\sffamily

\ifbachelor Bachelor \fi\ifmaster Master \fi\ifdiploma Diploma \fi Thesis\\[5pt]
\textbf{\firstname{} \lastname{}}
\vskip 1cm

The present work was submitted to the\\[5pt]
\textbf{Chair of Communication and Distributed Systems\\[5pt]
        RWTH Aachen University, Germany}
\vskip 2cm

Advisor(s):
\vskip 2mm
\advisers{}
\vskip 5mm
Examiners:
\vskip 2mm
\examiners{}
\vskip 1cm

\begin{tabular}{R{6cm}p{6cm}}
Registration date:  & \DTMusedate{RegistrationDate} \\
Submission date:    & \DTMusedate{SubmissionDate} \\
\end{tabular}

} %\large\sffamily
\fi

\ifphd
\ifphdsubmitted
Der Fakult\"at f\"ur Mathematik, Informatik und Naturwissenschaften\\der RWTH Aachen University vorgelegte Dissertation zur Erlangung\\des akademischen Grades \aimeddegreeGermanPossesive{}
\fi
\ifphdapproved
Von der Fakult\"at f\"ur Mathematik, Informatik und Naturwissenschaften\\der RWTH Aachen University zur Erlangung des akademischen Grades\\\aimeddegreeGermanPossesive{} genehmigte Dissertation
\fi

\vspace{2cm}
\ifphdsubmitted
von
\fi
\ifphdapproved
vorgelegt von
\fi

\vspace{2cm}
\currentdegree{}\\
\vspace{1em}
{\large{\textbf{\firstname{} \lastname{}}}}\\
\vspace{1em}
aus \placeofbirth{}

\ifphdsubmitted
\vspace{7cm}
\fi

\ifphdapproved
\vspace{2cm}
Berichter:\\
\vspace{1em}
\examiners{}\\
\vspace{2cm}
Tag der m\"undlichen Pr\"ufung: {\DTMsetup{datesep=.}\DTMsetstyle{ddmmyyyy}\DTMusedate{DefenseDate}}
\fi
\fi

\vfill % vertically center text on page

\end{titlepage}


\cleardoublepage
\ifundergrad
% !TEX root = ../thesis.tex

\ifbachelor
\includepdf[pages={1},offset=75 -75,pagecommand={
\begin{tikzpicture}[remember picture,overlay,shift={(current page.center)}]
\coordinate (de_begin-de_generic) at (5.4,10.3);
\coordinate (de_end-de_generic) at (6.4,10.3);
\coordinate (de_begin-de_master) at (-5.35,9.8);
\coordinate (de_end-de_master) at (-3.2,9.8);
\coordinate (en_begin-en_generic) at (3.2,9.35);
\coordinate (en_end-en_generic) at (3.9,9.35);
\coordinate (en_begin-en_master) at (6.0,9.35);
\coordinate (en_end-en_master) at (7.65,9.35);
\node at (-5.5,12.1)
       [text width=7cm,rounded corners,below right]
{
\mbox{\lastname{}, \firstname{}}
};
\node at (3.25,12.1)
       [text width=5cm,rounded corners,below right]
{
\studentid{}
};
\draw[line width=1mm] (de_begin-de_generic) -- (de_end-de_generic);
\draw[line width=1mm] (de_begin-de_master) -- (de_end-de_master);
\draw[line width=0.5mm] (en_begin-en_generic) -- (en_end-en_generic);
\draw[line width=0.5mm] (en_begin-en_master) -- (en_end-en_master);
\node at (-5.5,8.95)
       [text width=16.5cm,rounded corners,below right]
{
\begin{onehalfspacing}%
\thesistitleA\phantom{g}\\[0.1em]
\thesistitleB\phantom{g}\\[0.1em]
\thesistitleC\phantom{g}\\[0.1em]
\end{onehalfspacing}
};
\node at (-5.5,2.35)
       [text width=7cm,rounded corners,below right]
{
Aachen, {\DTMsetup{datesep=.}\DTMsetstyle{ddmmyyyy}\DTMusedate{SubmissionDate}}
};
\node at (-5.5,-8.7)
       [text width=7cm,rounded corners,below right]
{
Aachen, {\DTMsetup{datesep=.}\DTMsetstyle{ddmmyyyy}\DTMusedate{SubmissionDate}}
};
\end{tikzpicture}
}]{Formular_Eidesstattliche_Versicherung_neu.pdf}
\fi

\ifmaster
\includepdf[pages={1},offset=75 -75,pagecommand={
\begin{tikzpicture}[remember picture,overlay,shift={(current page.center)}]
\coordinate (de_begin-de_generic) at (5.4,10.3);
\coordinate (de_end-de_generic) at (6.4,10.3);
\coordinate (de_begin-de_bachelor) at (6.5,10.3);
\coordinate (de_end-de_bachelor) at (9,10.3);
\coordinate (en_begin-en_generic) at (3.2,9.35);
\coordinate (en_end-en_generic) at (3.9,9.35);
\coordinate (en_begin-en_bachelor) at (4.0,9.35);
\coordinate (en_end-en_bachelor) at (5.9,9.35);
\node at (-5.5,12.1)
       [text width=7cm,rounded corners,below right]
{
\mbox{\lastname{}, \firstname{}}
};
\node at (3.25,12.1)
       [text width=5cm,rounded corners,below right]
{
\studentid{}
};
\draw[line width=1mm] (de_begin-de_generic) -- (de_end-de_generic);
\draw[line width=1mm] (de_begin-de_bachelor) -- (de_end-de_bachelor);
\draw[line width=0.5mm] (en_begin-en_generic) -- (en_end-en_generic);
\draw[line width=0.5mm] (en_begin-en_bachelor) -- (en_end-en_bachelor);
\node at (-5.5,8.95)
       [text width=16.5cm,rounded corners,below right]
{
\begin{onehalfspacing}%
\thesistitleA\phantom{g}\\[0.1em]
\thesistitleB\phantom{g}\\[0.1em]
\thesistitleC\phantom{g}\\[0.1em]
\end{onehalfspacing}
};
\node at (-5.5,2.35)
       [text width=7cm,rounded corners,below right]
{
Aachen, {\DTMsetup{datesep=.}\DTMsetstyle{ddmmyyyy}\DTMusedate{SubmissionDate}}
};
\node at (-5.5,-8.7)
       [text width=7cm,rounded corners,below right]
{
Aachen, {\DTMsetup{datesep=.}\DTMsetstyle{ddmmyyyy}\DTMusedate{SubmissionDate}}
};
\end{tikzpicture}
}]{Formular_Eidesstattliche_Versicherung_neu.pdf}
\fi

\ifdiploma
\includepdf[pages={1},offset=75 -75,pagecommand={
\begin{tikzpicture}[remember picture,overlay,shift={(current page.center)}]
\coordinate (de_begin-de_bachelor) at (6.5,10.3);
\coordinate (de_end-de_bachelor) at (9,10.3);
\coordinate (de_begin-de_master) at (-5.35,9.8);
\coordinate (de_end-de_master) at (-3.2,9.8);
\coordinate (en_begin-en_bachelor) at (4.0,9.35);
\coordinate (en_end-en_bachelor) at (5.9,9.35);
\coordinate (en_begin-en_master) at (6.0,9.35);
\coordinate (en_end-en_master) at (7.65,9.35);
\node at (-5.5,12.1)
       [text width=7cm,rounded corners,below right]
{
\mbox{\lastname{},\firstname{}}
};
\node at (3.25,12.1)
       [text width=5cm,rounded corners,below right]
{
\studentid{}
};
\draw[line width=1mm] (de_begin-de_bachelor) -- (de_end-de_bachelor);
\draw[line width=1mm] (de_begin-de_master) -- (de_end-de_master);
\draw[line width=0.5mm] (en_begin-en_bachelor) -- (en_end-en_bachelor);
\draw[line width=0.5mm] (en_begin-en_master) -- (en_end-en_master);
\node at (-5.5,8.95)
       [text width=16.5cm,rounded corners,below right]
{
\begin{onehalfspacing}%
\thesistitleA\phantom{g}\\[0.1em]
\thesistitleB\phantom{g}\\[0.1em]
\thesistitleC\phantom{g}\\[0.1em]
\end{onehalfspacing}
};
\node at (-5.5,2.35)
       [text width=7cm,rounded corners,below right]
{
Aachen, {\DTMsetup{datesep=.}\DTMsetstyle{ddmmyyyy}\DTMusedate{SubmissionDate}}
};
\node at (-5.5,-8.7)
       [text width=7cm,rounded corners,below right]
{
Aachen, {\DTMsetup{datesep=.}\DTMsetstyle{ddmmyyyy}\DTMusedate{SubmissionDate}}
};
\end{tikzpicture}
}]{Formular_Eidesstattliche_Versicherung_neu.pdf}
\fi

\cleardoublepage
\fi
% !TEX root = ../thesis.tex

\pdfbookmark[0]{Abstract}{Abstract}

\begin{center}
\paragraph{Abstract}
\hrulefill
\end{center}
Außerhalb der Umgebung hingegen erfolgt die Trennung des Donaudampfschifffahrtskapitäns nicht korrekt.


\ifphd
\clearpage % PhD thesis abstracts are commonly about a page...
\fi
\ifundergrad
\vspace {2cm} % ... while undergrad thesis abstracts are commonly about half a page
\fi
% !TEX root = ../thesis.tex

\pdfbookmark[0]{Kurzfassung}{Kurzfassung}

\begin{otherlanguage*}{ngerman}
\begin{center}
\paragraph{Kurzfassung}
\hrulefill
\end{center}
In der \texttt{otherlanguage*}-Umgebung erfolgt die Trennung des Donaudampfschifffahrtskapitäns korrekt.


\end{otherlanguage*}

\cleardoublepage
\ifphdapproved
% !TEX root = ../thesis.tex

\pdfbookmark[0]{Acknowledgments}{Acknowledgments}

\chapter*{Acknowledgments}

Don't forget to thank your dog


\cleardoublepage
\fi
\ifphdsubmitted
% !TEX root = ../thesis.tex

\section*{Declaration of Authorship}
\pdfbookmark[0]{Declaration of Authorship}{Declaration of Authorship}

I, \firstname{} \lastname{}

declare that this thesis and the work presented in it are my own and has been generated by me as the result of my own original research.

I do solemnely swear that:

\begin{enumerate}[label=\arabic*.,topsep=0pt]
	\item This work was done wholly or mainly while in candidature for the doctoral degree at this faculty and university;
	\item Where any part of this thesis has previously been submitted for a degree or any other qualification at this university or any other institution, this has been clearly stated;
	\item Where I have consulted the published work of others or myself, this is always clearly attributed;
	\item Where I have quoted from the work of others or myself, the source is always given.
		This thesis is entirely my own work, with the exception of such quotations;
	\item I have acknowledged all major sources of assistance;
	\item Where the thesis is based on work done by myself jointly with others, I have made clear exactly what was done by others and what I have contributed myself;
	\item Parts of this work have been published before.
		A detailed list can be found on the following page.
\end{enumerate}

\vspace{20pt}
{\DTMsetup{datesep=.}\DTMsetstyle{en-US}\DTMusedate{SubmissionDate}}

\newpage
\section*{Published Papers}
\pdfbookmark[0]{Published Papers}{Published Papers}

Parts of this thesis are based on the following peer-reviewed papers that have already been published.
All my collaborators are among my co-authors.
A detailed attribution of contributions can be found in Section~\ref{...} on page~\pageref{...}.

\subsection*{List of Publications}

\nociteown{*}
\begingroup
	\def\chapter*#1{}
	\bibliographyown{literature/own}
	\bibliographystyleown{styles/comsys-alpha}
\endgroup


\cleardoublepage
\fi

% Titelseite hatte noch normale Tabellen. Von hier ab sollen alle
% Tabellen laut style-Vorgaben sans serif sein.
\AtBeginEnvironment{tabular}{\sffamily}
\AtBeginEnvironment{tabularx}{\sffamily}

%% Inhaltsverzeichnis %%%%%%%%%%%%%%%%%%%%%%%%%%%%%%%%%%%%%%%
\setcounter{tocdepth}{2}
\tableofcontents %Inhaltsverzeichnis
\cleardoublepage %Das erste Kapitel soll auf einer ungeraden Seite beginnen.
} % end ifnotdraft

\pagestyle{fancy} %%Ab hier die Kopf-/Fusszeilen: headings / fancy / ...

%%%%%%%%%%%%%%%%%%%%%%%%%%%%%%%%%%%%%%%%%%%%%%%%%%%%%%%%%%%%%
% einzelne Kapitel
%%%%%%%%%%%%%%%%%%%%%%%%%%%%%%%%%%%%%%%%%%%%%%%%%%%%%%%%%%%%%
%\include{commands}

\mainmatter
\include{chapters/01_introduction}

%%%%%%%%%%%%%%%%%%%%%%%%%%%%%%%%%%%%%%%%%%%%%%%%%%%%%%%%%%%%%
%% LITERATUR UND ANDERE VERZEICHNISSE
%%%%%%%%%%%%%%%%%%%%%%%%%%%%%%%%%%%%%%%%%%%%%%%%%%%%%%%%%%%%%
%% Ein kleiner Abstand zu den Kapiteln im Inhaltsverzeichnis (toc)
\ifnotdraft{
\addtocontents{toc}{\protect\vspace*{\baselineskip}}
\cleardoublepage
%% Literaturverzeichnis
\phantomsection % phantomsection wird benötigt, damit z.B. hyperref die richtige Seite verlinkt.
\addcontentsline{toc}{chapter}{Bibliography}
\begin{footnotesize}
%\nocite{*} %Auch nicht-zitierte BibTeX-Einträge werden angezeigt.
\bibliography{literature/literature}%Eine Datei 'literature.bib' wird hierfür benötigt.
\bibliographystyle{styles/comsys-alpha}%Art der Ausgabe: plain / apalike / amsalpha / ...
\end{footnotesize}
}

%% Abbildungsverzeichnis
%\clearpage
%\addcontentsline{toc}{chapter}{List of Figures}
%\listoffigures

%% Tabellenverzeichnis
%\clearpage
%\addcontentsline{toc}{chapter}{List of Tables}
%\listoftables


%%%%%%%%%%%%%%%%%%%%%%%%%%%%%%%%%%%%%%%%%%%%%%%%%%%%%%%%%%%%%
%% ANHÄNGE
%%%%%%%%%%%%%%%%%%%%%%%%%%%%%%%%%%%%%%%%%%%%%%%%%%%%%%%%%%%%%
\appendix
% !TEX root = ../thesis.tex

\chapter{Appendix}

\printacronyms[heading=section,name=List of Abbreviations]{}



\end{document}
